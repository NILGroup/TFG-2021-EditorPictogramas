\chapter*{Abstract}

People with communication difficulties, whether due to cognitive or physical disability, need external aids to facilitate communication with their environment. The most common way to facilitate communication for these groups is the use of pictograms. These are mainly used on communication boards, although a wide variety of pictographic material can also be created, such as diaries, calendars or rulers.

With the technological advances of the last decades, a multitude of tools have emerged that allow the creation of pictographic material, but many of these tools are often limited to a specific format or do not offer many possibilities for customisation.  Therefore, there is a need for an application that integrates the options most used and demanded by users. 

The aim of this work is to develop a web application, called PictUp!, which allows teachers, parents and tutors to create pictographic material quickly and easily. The application is designed as a web application to reach as many users as possible. PictUp! consists of a grid board that allows users to place and precisely move the different elements of the grid at their convenience. In addition, it is included an extensive set of tools for working with pictograms and other elements, such as photos, text and icons.

Once the application was developed, an asynchronous evaluation was carried out with a user group consisting of both people who create and use pictographic material and people who are not involved in the pictographic field. After an analysis of the answers obtained in the questionnaire, it could be concluded that the application is easy to use and useful for the users.

\section*{Keywords}

\noindent Pictograms, Web application, Accessibility, Communication board, Pictographic material, Cognitive disability.



