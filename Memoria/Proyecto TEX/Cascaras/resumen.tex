\chapter*{Resumen}

Las personas con dificultades de comunicación, ya sea por discapacidad cognitiva o física, requieren de ayudas externas para facilitar la comunicación con su entorno. La manera más común que facilita la comunicación a estos colectivos es mediante  pictogramas. Estos son utilizados principalmente en tableros de comunicación, aunque también se puede crear una gran variedad de material pictográfico como agendas, calendarios o normas.

Con los avances tecnológicos de las últimas décadas surgieron multitud de herramientas que permiten la creación de materiales pictográficos para estas personas, pero muchas de estas herramientas suelen estar limitadas a un formato concreto de material o no ofrecen muchas posibilidades de personalización. Por ello, surge la necesidad de una aplicación que integre las opciones más utilizadas y demandadas por los usuarios. 

El objetivo de este trabajo es el de desarrollar una aplicación web, llamada PictUp!, que facilite a profesores, padres y tutores crear material pictográfico de manera rápida y sencilla. La aplicación está planteada como aplicación web para alcanzar al mayor número de usuarios posible. PictUp! consta de un tablero cuadriculado que permite a los usuarios desplazar sobre él los distintos elementos disponibles con precisión, para colocarlos según su conveniencia. Además, se incluye un amplio conjunto de herramientas para trabajar con pictogramas y otros elementos, como fotos, texto e iconos.

Una vez desarrollada la aplicación, se realizó una evaluación asíncrona con un grupo de usuarios compuesto tanto por personas que crean y utilizan material pictográfico como por otras ajenas al tema. Tras un análisis de las respuestas obtenidas en el cuestionario, se pudo concluir que la aplicación es fácil de usar, además de ser de gran utilidad para los usuarios.


\section*{Palabras clave}
   
\noindent Pictogramas, Aplicación Web, Accesibilidad, Tablero de comunicación, Material pictográfico, Discapacidad cognitiva. 

   


