\chapter{PictUp}
\label{cap:introduccion}

%\chapterquote{}{}

\begin{resumen}
	El nombre es provisional
\end{resumen}

\label{cap1:sec:Motivacion}


\section{Introducción}

\section{Prototipado de la interfaz}

\section{Componentes}


En esta sección desarrollaremos los principales componentes que pueden colocarse sobre el tablero. 

Un componente es \textbf{un elemento que puede ser añadido al tablero} e incluso añadir interacción para al usuario final. Distinguiremos dos tipos de componentes, los \textbf{básicos} que no añaden ninguna interacción y los \textbf{componentes interactivos}, que pueden ser modificados y añadir comportamientos específicos al usuario final.

\begin{itemize}
	
	
\item Picto: El elemento picto representa un pictograma junto a su nombre asociado. Cuenta con la opción de poder modificar algunas características del pictograma, como indicar un tiempo verbal, el plural. Si el pictograma muestra a una persona, también se puede cambiar el color de pelo y tono de piel.

\item Foto: El elemento foto, permite añadir imágenes tanto a partir de una URL como las que suba el propio usuario. Esta ha sido una de las características más demandadas por los usuarios. Al permitir añadir fotos abre multitud de posibilidades, como la de poner fotos de la familia, mostrar localizaciones habituales como la cocina e identificar objetos personales que no se representan tan fielmente mediante un pictograma (Por ejemplo, un juguete específico o la portada de su  libro favorito). Esto facilita al usuario final relacionar conceptos al mostrar figuras que le sean familiares.

\item Figuras: Las figuras, sirven para ordenar, enfatizar o decorar el tablero. Un ejemplo, sería una línea que puede ser usada para dividir el espacio de trabajo en secciones, relacionar dos pictogramas o incluso marcar un espacio donde escribir una respuesta si se va a imprimir el tablero. Pese a la simplicidad de las figuras, sus posibilidades son muy amplias, según  la creatividad de quien cree el tablero.


\item Colecciones: Las colecciones son conjuntos de pictogramas creadas por el usuario. La finalidad de las colecciones son las de poder acceder con rapidez a pictogramas recurrentes que pueda necesitar el usuario. Las colecciones pueden ser de utilidad por ejemplo, si un profesor tiene que realizar varios materiales sobre un tema en concreto. Evitando así tener que buscar repetidamente el mismo pictograma. 



\item Cajón de Pictogramas: El cajón de pictos es un apartado al margen del tablero donde aparecen un conjunto de pictogramas que el usuario debe mover a alguna posición. En el hueco donde vayan los pictogramas que se encuentren en el cajón de pictogramas puede ser modificado y aceptar unos u otros. Esto puede ser utilizado como test sencillo de hacer y usar.

\item Subtablero: El Subtablero es un componente que a simple vista parece un pictograma pero al ser pulsado, despliega un tablero que contiene otros pictogramas. Este concepto la ha sido rescatada de Piktoplus, la cual actualmente no cuenta con soporte y puede ser de utilidad para añadir más pictogramas en el mismo espacio.
\end{itemize}
