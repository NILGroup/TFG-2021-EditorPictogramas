\chapter{Tecnologías}
\label{cap:introduccion}

%\chapterquote{}{}

\begin{resumen}
	
\end{resumen}

\label{cap1:sec:Motivacion}


\section{Introducción}

\section{API Arasaac}

\section{React}

\section{Drag and Drop}

React Drag and Drop\footnote{\url{https://react-dnd.github.io/react-dnd/about}} Es una biblioteca que permite arrastrar componentes en React. Su principal uso en la aplicación, es la recolocación de los componentes del tablero de manera cuadriculada. Esto permite colocar con precisión los elementos en el tablero, lo cual era un requisito indispensable para poder hacer tableros organizados. De otra manera podrían quedar algunos componentes unos pocos píxeles por encima de otro, dando un resultado poco profesional.

\section{JSZip}

JSZip \footnote{\url{https://stuk.github.io/jszip/}} es una biblioteca de javascript que mediante una API permite crear y cargar archivos comprimidos en formato \textit{ZIP}. Será el método para importar y exportar proyectos a la aplicación.
\begin{itemize}
	\item \textbf{Generar ZIP}: Crea un zip con todo lo que haya creado el usuario, como las fotos que haya subido, la posición de los pictogramas colocados y su información asociada (Por ejemplo, si se ha cambiado la descripción de un pictograma)
	\item \textbf{Cargar ZIP}: Al subir un ZIP, vuelve al estado cuando fue generado.
\end{itemize}	