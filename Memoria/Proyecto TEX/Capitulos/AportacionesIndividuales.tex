\chapter{Aportaciones individuales al proyecto}
\label{cap:Aportacionesindividuales}

\section{Alfonso Tercero López}

Gran parte del trabajo fue realizado junto a mi compañero Jorge. A continuación comentaré las secciones que he realizado de manera individual o me ha ayudado a mi compañero.

El capítulo de Introducción fue escrito de manera conjunta entre Jorge y yo. 

Se realizó de manera conjunta la investigación de los distintos tipos de sistemas pictográficos así como las distintas aplicaciones. Concretamente desarrollé en la Memoria los apartados relacionados con SPC (Sección \ref{cap3:sec:spc}), Blissymbolics (Sección \ref{cap3:sec:Blissymbolics}), Mulberry Symbols (Sección \ref{cap3:sec:mulberry}) y Minspeak (Sección \ref{cap3:sec:minspeak}). Respecto a las aplicaciones, escribí sobre Pictoselector (Sección \ref{cap2:sec:pictoselector}), Piktoplus (Sección \ref{cap2:pkplus}) y SymboTalk (Sección \ref{cap2:sec:symbotalk}). Ayudé a mi compañero a comparar las distintas herramientas y discutimos sobre sus diferencias.

Respecto a las tecnologías, realicé varios prototipos tecnológicos en distintas tecnologías para conocer sus características y posibilidades. Esto aparece reflejado en la memoria en la API de ARASAAC(Sección \ref{cap3:sec:apiarasaac}), JSZip( Sección \ref{cap3:sec:jszip}) y Progressive Web Application (Sección \ref{cap3:sec:pwa})

Una vez vista las tecnologías, también realicé bocetos de la aplicación en papel para compararlos con mi compañero y sacar conclusiones. El prototipo puede verse en la Sección \ref{cap4:sec:alfonso}. A su vez Jorge y yo establecimos los distintos requisitos de la aplicación. 

De cara al desarrollo de la aplicación desarrollé las funcionalidades referentes a la búsqueda de pictogramas(Sección \ref{cap5:buscador}), traducción de frases a pictogramas (Sección \ref{cap5:sec:traduccion}) y añadir imágen al tablero (Sección \ref{cap5:sec:imagen}). Estas contaron con el apoyo de mi compañero para corregir errores y unificar la interfaz de estas funcionalidades. Escribí la parte correspondiente a la arquitectura (Seccion \ref{cap5:arquitectura}) tras ser estudiada y comprendida por mi compañero y yo. Buena parte del tiempo de desarrollo fue dedicada a otras funcionalidades que no fueron implementadas, como la conexión con la API de Google Drive y  de NIL-WS-API, que desembocó en un backend que no se utilizó. Esto queda reflejado en la memoria en la Sección \ref{cap5:noimplementadas}. Por último también gestioné el despliegue de la aplicación (Sección \ref{cap5:despliegue}). 

Durante el período comprendido entre el desarrollo de la aplicación y la evaluación con usuarios, gestioné la cuenta de Instagram. Para ella produje algunos vídeos que explicaban su funcionamiento con la intención de dar a conocer la aplicación y captar usuarios para la evaluación. 

Respecto al capítulo de la evaluación trabajé de forma conjunta con Jorge para diseñar el formulario y realizar el análisis. De la memoria escribí la preparación de la evaluación (Sección \ref{eva:prep}) y la las observaciones de los usuarios (Sección \ref{cap6:sec:observaciones}). También ayudé a mi compañero a escribir el resultado de la evaluación. 

Respecto al capítulo siete, escribí las conclusiones del trabajo futuro (Sección \ref{cap7:sec:conclusiones}).



\section{Jorge García Cerros}

A continuación, detallaré todos los apartados que he escrito de manera individual o he ayudado al desarrollo de estos junto con mi compañero Alfonso.


Tras acordar la realización del Trabajo de Fin de Grado con los directores Raquel Hervás y Gonzalo Méndez, nos dedicamos tanto Alfonso como yo, a desarrollar el capítulo de Introducción.


La búsqueda de información sobre los distintos tipos de sistemas pictográficos y búsqueda de aplicaciones se hizo de manera conjunta. Yo me encargué de escribir lo referente a los Sistemas Aumentativos y Alternativos (Sección \ref{cap3:sec:saac}) así como los pictogramas Sclera (Sección \ref{cap2:sec:sclera}) y ARASAAC (Sección \ref{cap3:sec:arasaac}) y distintas aplicaciones encontradas como Editor ARASAAC (Sección \ref{cap2:sec:editor}), Pictar (Sección \ref{cap2:sec:pictar}), PicTablero (Sección \ref{cap2:pictableros}), LetMe Talk (Sección \ref{cap2:sec:letmetalk}) y Jocomunico (Sección \ref{cap2:jocomunico}). También se analizó de manera conjunta las distintas aplicaciones para su posterior análisis en la Sección \ref{cap2:tablacomparativa}.


En cuanto al apartado de las tecnologías se realizaron de manera individual distintos prototipos con el fin de investigar sobre las distintas tecnologías existentes. El resultado del trabajo realizado queda reflejado en el apartado React (Sección \ref{cap3:sec:react}) y sus correspondientes subapartados Drag and Drop (Sección \ref{cap3:sec:draganddrop}) y Prototipos (Sección \ref{cap3:sec:prototipos}).


Tras haber estudiado las distintas tecnologías se acordó en la realización de unos bocetos en papel sobre la futura aplicación. Este prototipo realizado queda reflejado en la Sección \ref{cap4:sec:prototipo}. También se hizo de manera conjunta el apartado de Requisitos de la aplicación (Sección \ref{cap4:requisitosapp}). Además, investigué y escribí sobre las distintas reglas de diseño para realizar una correcta interfaz en la aplicación (Sección \ref{cap4:sec:reglas}).


Respecto a la parte del desarrollo de la aplicación me he encargado mayoritariamente del apartado visual. Para ello me encargué de la instalación de las librerías necesarias para la utilización de Bootstrap y FontAwesome. En lo referente a la maquetación he utilizado las clases definidas por Bootstrap además de la creación de hojas de estilo CSS para los distintos componentes de la aplicación. También me he encargado de la gran parte de las visualizaciones de las distintas componentes de la aplicación. Asimismo, he desarrollado los componentes: listas de pictogramas (Sección \ref{listapictos}), texto (Sección \ref{cap5:sec:texto}), iconos (Sección \ref{cap5:sec:iconos}) y descargar tablero (Sección \ref{cap5:descargar}). Por último escribí el control de versiones y organización del desarrollo de PictUp! (Sección \ref{cap5:controlversiones}).


Para la creación de la evaluación con los usuarios tanto Alfonso como yo nos encargamos de realizar los formularios y analizar los resultados. Del capítulo de evaluación de la memoria escribí el apartado de diseño de la evaluación (Sección \ref{eva:res}) y todas las tablas que aparecen en el análisis de los resultados (Sección \ref{cap6:sec:evaluacion}).


Por último en el capítulo 7 me encargue de escribir el Trabajo Futuro (Sección \ref{cap7:sec:trabajofuturo}).












