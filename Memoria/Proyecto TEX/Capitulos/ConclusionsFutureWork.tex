\chapter{Conclusions and Future Work}
\label{cap:conclusions}

\section{Conclusions}


The main objective that was proposed for the realization of this final degree project was to create a web application that would facilitate the creation of pictographic materials, focused on the creation of communication boards. To do this, we learned about the different types of pictographic systems and which people used to use them. After a long research, we saw that the pictographic system that had the biggest amount of pictograms was ARASAAC, so we decided to use their pictograms for the development of the application. In addition, the application was developed to cover certain gaps that other board creation applications did not have. These deficiencies used to be functionalities such as adding your own images to the board, being able to translate a sentence into pictograms, adding icons or the possibility to align easily and precisely the board elements.

After developing the application, an evaluation was carried out with users to obtain data regarding usability and usefulness. This evaluation was of great value, since it was possible to confirm that the usability was high. In addition, it was possible to contrast which functionalities had to be improved, in addition to knowing the opinions of the user about the existing ones.

Another proposed objective was to be able to demonstrate all the knowledge acquired during the degree and to expand it. The most influential subjects for the completion of the final degree project were:



\begin{itemize}
	\item Web applications: in this subject we acquired knowledge about HTML, JavaScript and layout. What we learned in JavaScript was of great help for the development of the application in React. 
	\item Development of interactive systems: this subject allowed us to learn about the different forms of evaluation on an application and to see the importance of these in order to detect errors and be solved.
	\item User Interfaces: closely linked to Interactive System Development, this subject allowed us to learn the design rules to follow to create intuitive interfaces in the application.
\end{itemize}



\section{Future work}

Once the development of the application and its evaluation were completed, we grouped the improvements and new functionalities that could not be implemented and that would be interesting to add in the future: 

\begin{itemize}
	\item Improve the functionality of translating a sentence to pictograms using NIL-WS-API.
	\item Allow the user to be able to save the state of the application through Google Drive to later reload the state of the board and upload images that are in their account.
	\item Implement interactive components, such as sub-boards or draggable pictograms that can be moved by the user to play a game.
	\item Offer the possibility to hide the grid that is drawn on the application's board.
	\item Offer the possibility to change the size of the board.
	\item Implement another type of board that allows an edition more similar to that offered in Power Point, where the elements on the board can be aligned more easily.  
	
	\item Improve the interface in the section of the pictogram lists and translation of sentence to pictograms for better usability.
	
	\item Add new icons to complement the pictograms, such as emoticons representing a happy or sad face.
	
	\item Offer the possibility of adding a video to the board and being able to play it from the board. This would help the user to associate a pictogram with an action, for example with the pictogram of jumping next to a gif of a person jumping.
	
	\item Include a button in the application to delete all the elements that are on the board.
	
	\item Add an image search engine within the website. This would help to search images quickly avoiding the user having to download and upload the photos to the web. 
	
	\item Possibility of assigning any color to the border and background of a pictogram. The use of colors can be different depending on the user, so the possibility of free choice should be offered. 
	
\end{itemize}








