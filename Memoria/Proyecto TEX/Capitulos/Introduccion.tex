\chapter{Introducción}
\label{cap:introduccion}

\chapterquote{La inteligencia es la habilidad de adaptarse a los cambios}{Stephen Hawking}

%\begin{resumen} En este capítulo se explicará la motivación de este trabajo (Sección %\ref{cap1:sec:Motivacion}), los objetivos que se quieren lograr (Sección %\ref{cap1:sec:Objetivos}) y la estructura de esta memoria (Sección \ref{cap1:sec:Estructura}). 
%\end{resumen}

\section{Motivación}
\label{cap1:sec:Motivacion}
Los seres humanos siempre hemos tenido la necesidad inherente de comunicarnos y la población con dificultades en el lenguaje oral no debe quedar excluida. Como es el caso de las personas con trastornos del espectro autista, parálisis cerebral, esclerosis múltiple o parkinson, entre otros. 

Para facilitar la comunicación de estos colectivos se utilizan medios de comunicación
alternativos al lenguaje oral como los sistemas pictográficos compuestos por pictogramas, los cuales son imágenes que representan ideas y conceptos.
La  manera más común de utilizar estos sistemas es mediante tableros de comunicación. Estos tableros son superficies  compuestas por una selección de pictogramas que permiten al usuario con dificultades de comunicación formar mensajes. Un ejemplo de tablero es el que vemos en la Figura  \ref{fig:tablerofisico}  con el cual el usuario puede señalar un pictograma con el dedo para indicar un objeto u acción. Aparte de en los tableros, los pictogramas pueden ser utilizados de otras muchas maneras, como calendarios, agendas, listas de normas, etc.


% TODO: \usepackage{graphicx} required
\begin{figure}[h!]
	\centering
	\includegraphics[width=0.7\linewidth]{Imagenes/Bitmap/tablerofisico}
	\caption{Tablero pictográfico en el que el usuario señala lo que quiere comunicar.}
	\label{fig:tablerofisico}
\end{figure}


Los tableros de comunicación a menudo eran creados a mano, recortando y pegando los pictogramas en una cartulina. Con el tiempo se implementaron soluciones tecnológicas enfocadas a facilitar esta tarea. 

Existen multitud de aplicaciones que permiten crear material, pero generalmente están limitadas a un formato concreto y ofrecen poca libertad al usuario para crear material. Además cada una de estas aplicaciones cuentan con opciones y facilidades diferentes, como traducir una frase a pictogramas, un tablero donde se puedan colocar los pictogramas donde se desee, añadir figuras al tablero, etc. Pero no existe ninguna aplicación que englobe todas estas opciones. 

Es por ello que la finalidad de este trabajo es la de crear una aplicación web que permita crear material pictográfico de manera rápida y sencilla, ofreciendo la mayor libertad posible al usuario junto a las opciones más utilizadas y demandadas por los usuarios en una única herramienta. 





\section{Objetivos}
\label{cap1:sec:Objetivos}


El objetivo del trabajo es el de desarrollar una aplicación web que permita a los usuarios crear material pictográfico reuniendo las funcionalidades más utilizadas y pedidas por los usuarios. La aplicación debe contar con un tablero que permita desplazar los pictogramas y otros elementos con precisión.

Para ello, se investigarán las distintas aplicaciones existentes y cuáles son las funcionalidades más reclamadas por los usuarios. También se investigarán y ampliarán conocimientos sobre distintas tecnologías web actuales. 

Otro objetivo es la de crear una interfaz sencilla e intuitiva para el usuario, y pueda ser utilizada en el mayor número de dispositivos posible. Para contrastar los objetivos del trabajo, se realizará una evaluación donde se compruebe la usabilidad y la facilidad de uso de la aplicación. 








\section{Estructura de la memoria}
\label{cap1:sec:Estructura}

La estructura para memoria se encuentra dividida en 8 capítulos, a continuación se explicará brevemente su contenido. 
\begin{itemize}
	\item En los capítulos uno y dos se expondrá el contexto bajo el cual se ha realizado el trabajo junto a la motivación y objetivos para realizarlo.
	
	\item En el capítulo tres se explicará qué es un pictograma y los distintos sistemas de comunicación basados en ellos. Además se analizarán las distintas herramientas relacionadas con pictogramas haciendo énfasis en la edición de tableros.
	
	\item En el capítulo cuatro se especificarán los requisitos y explicarán los prototipos creados.
	
	\item En el capítulo cinco se explicará en detalle la arquitectura y diseño de la aplicación.
	
	\item El capítulo seis mostrará el diseño, resultado y análisis de la evaluación realizada por los usuarios. 
	
	\item En el capítulo siete, se presentarán las conclusiones finales y se especificará el trabajo futuro a realizar.
	
	\item En el capítulo ocho se detallarán las tareas realizadas de los dos integrantes del trabajo.
\end{itemize}	




