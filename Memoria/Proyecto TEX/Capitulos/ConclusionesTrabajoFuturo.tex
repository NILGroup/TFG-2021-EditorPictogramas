\chapter{Conclusiones y Trabajo Futuro}
\label{cap:conclusiones}

\section{Conclusiones}
\label{cap7:sec:conclusiones}

El principal objetivo que se propuso de cara a la realización de este trabajo de fin de grado fue el poder realizar una aplicación web que facilitara la creación de materiales pictográficos, enfocado a la creación de tableros. Para ello nos informamos sobre los distintos tipos de sistemas pictográficos y qué personas solían utilizarlos. Tras una larga búsqueda vimos que el sistema pictográfico que mayor material proporcionaba era ARASAAC por lo que se decidió utilizar sus pictogramas para el desarrollo de la aplicación. Además la aplicación se desarrolló para cubrir ciertas carencias que otras aplicaciones de creación de tableros no tenían. Estas carencias solían ser funcionalidades como añadir imágenes propias al tablero, poder traducir una frase a pictogramas, añadir iconos o tener la posibilidad de tener los elementos del tablero fácilmente alineados. 

Tras desarrollar la aplicación se realizó una evaluación con usuarios para obtener datos referentes a la usabilidad y utilidad. Esta evaluación resultó de gran valor, ya que se pudo afirmar que la usabilidad es alta. Además se pudo contrastar qué funcionalidades se tenían que mejorar, además de conocer las opiniones de los usuarios, sobre los ya existentes.


Otro objetivo propuesto fue poder demostrar todos los conocimientos adquiridos durante la carrera y ampliarlos. Las asignaturas más influyentes para la realización del trabajo fin de grado fueron:

\begin{itemize}
	\item \textbf{Aplicaciones Web}: en esta asignatura se adquirieron conocimientos sobre HTML, Javascript y maquetación. Lo aprendido en JavaScript resultó de gran ayuda de cara al desarrollo de la aplicación en React. 
	
	\item\textbf{ Desarrollo de Sistemas Interactivos}: esta asignatura nos permitió conocer las diferentes formas de evaluación sobre una aplicación y ver la importancia de estas para detectar errores y ser solucionados.
	
	\item \textbf{Interfaces de usuario}: muy ligado al Desarrollo de Sistema Interactivos, esta asignatura nos permitió conocer reglas de diseño a seguir para crear interfaces intuitivas en la aplicación.
	
\end{itemize}


\section{Trabajo futuro}
\label{cap7:sec:trabajofuturo}
Una vez terminado el desarrollo de la aplicación y la evaluación de la misma, se recogieron las mejoras y nuevas funcionalidades que no pudieron ser implementadas y que sería interesante añadir en un futuro: 

\begin{itemize}
	\item Mejorar la funcionalidad de traducción de una frase a pictogramas mediante NIL-WS-API.
	\item Permitir al usuario poder guardar el estado de la aplicación por medio de Google Drive para posteriormente volver a cargar el estado del tablero y cargar imágenes que se encuentren en su cuenta.
	
	\item Implementar componentes interactivos, como subtableros o cajón de pictos.
	
	\item Ofrecer la posibilidad de poder ocultar la cuadrícula que se dibuja el tablero de la aplicación.
	\item Ofrecer la posibilidad de cambiar el tamaño del tablero.
	
	\item Implementar otro tipo de tablero que permita una edición más semejante a la ofrecida en Power Point, donde los elementos sobre el tablero puedan ser alineados con mayor facilidad.  
	\item Mejorar la interfaz en el apartado de las listas de pictogramas y traducción de frase a pictogramas para que tengan una mejor usabilidad.
	
	\item Añadir nuevos iconos que complementen a los pictogramas, como emoticonos que representen una cara feliz o triste.
	
	\item Ofrecer la posibilidad de añadir un vídeo al tablero y poder reproducirlo desde el mismo. Esto ayudaría al usuario a asociar un pictograma con una acción, por ejemplo con el pictograma de saltar junto a un gif de una persona saltando.
	
	\item Incluir un botón en la aplicación para borrar todos los elementos que estén en el tablero.
	
	\item Añadir un buscador de imágenes dentro de la web. Esto ayudaría a buscar imágenes de manera rápida evitando que el usuario tenga que descargar y subir las fotos a la web. 
	
	\item Posibilidad de asignar cualquier color al borde y fondo de un pictograma. El uso de los colores puede ser distinto dependiendo del usuario, por lo que se debe ofrecer la posibilidad de elegirlo libremente. 
	
\end{itemize}








